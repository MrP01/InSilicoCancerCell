\section{Introduction}
The mechanisms behind lung adenocarcinoma, specifically those of individual A549 cells.
Computational techniques can help with a better understanding of the behaviour of these cancer cells.
% , together with a calcium channel extension introduced in \cite{2024-calcium-channels}, 
We work with the A549 model introduced in \cite{2021-A549-model,2024-calcium-channels}, reimplementing the model in the Rust programming language and performing a number of numerical optimizations such as adaptive timestepping.
Physical measurements are obtained using a \textit{Patch-Clamp System}, where one records the current through the membrane given a voltage protocol.
% We also verify the model's performance in \textit{Floating} mode, as compared to an individual simulation of the estimated number of channels.
In order to find appropriate parameters for the model, an optimization procedure is performed.
Multiple optimization approaches for the solution of the corresponding inverse problem (fitting model parameters to measurement data) are put in comparison.

\section{Methods}
A single cell's membrane consists of multiple ion channels, categorized into $M \in \N$ different types.
Each ion channel is represented in one of $N_{s, k} \in \N$ states, which, in physical terms, is related to a positional configuration of a protein within the ion channel.
Only some states can be observed directly, and its development only depends on the one previous state.
Hence, we are working with a Hidden Markov Model (HMM).
% where transitions between states are modelled probabilistically.
For many ion channel categories, their transition probabilities are voltage or ion-concentration dependent.

The whole cell current $I: T \to \R$ over time $t \in T \subset \R^+$ is then obtained as the sum of all individual channel contributions $I_k, k \in \{1, ..., M\}$ over $M \in \N$ channel types
\begin{equation}
  I(t) := \sum_{k=1}^{M} N_k I_k(t) = \sum_{k=1}^{M} N_k g_k p_{o,k} \left(V(t)-E_k\right)\,,
  \label{eq:current}
\end{equation}
where $N_k$ is the number of channels of type $k \in \{1, ..., M\}$, $g_k$ is the respective ion channel's conductivity, $p_{o, k} \in [0, 1]$ is the probability of observing the channel in a state where an ion current can flow (``open states''), $V: T \to \R$ is the voltage across the membrane and $E_k \in \R$ the reversal potential.

Within the simulation, we sample the state and current at discrete time points $T_{\rm meas} \subset T$, for example
$$T_{\rm meas} := \left\{t_n := \sum_{i=0}^n (\Delta t)_i \;\bigg|\; n \in \N_0 \;\bigg|\; n < N_t\right\}$$
for $N_t$ measurements with step size $(\Delta t)_n$, which may be chosen equally large for all $n \in \{0, ..., N_t - 1\}$.
We adapt this time interval $(\Delta t)_n \in \R^+$ per simulation step based on a state change heuristic, cf. \Cref{sec:adaptive-dt}.

At each time step,
\begin{equation}
  \vec{s}_{k,n+1} = H_{k}\left(V(t_n), \vec{C}(t_n), t_n\right) \vec{s}_{k,n}
\end{equation}
where $\vec{s}_{k,n} \in [0, 1]^{N_{s,k}}$ is the state vector of ion channel type $k$ at the $n$-th time step, $H_{k}\left(V, \vec{C}, t_n\right) \in [0, 1]^{N_{s,k} \times N_{s,k}}$ the transition matrix for type $k$ with $\sum_{j=1}^{N_{s,k}} \{H_k\}_{i,j} = 1 \;\forall\,i$, $V(t_n)$ the voltage across the membrane at time $t_n$ and $\vec{C}(t_n) \in \R^4$ the concentrations of Kalium, Calcium, Sodium and Chlorine at time $t_n$.
We initialize the simulation at $t_0 = 0$ with $\vec{s}_{k,0} = (1, 0, ..., 0)^T$ for all $k$.

Given the state $\vec{s}_{k}$, current measurements are then simply
$$\vec{I} := \left(I(t_0), I(t_1), ..., I(t_{N_t-1})\right)^T \in \R^{N_t}\,,$$
with $I$ as stated above in \Cref{eq:current} and
$$p_{o,k} = \sum_{j \in \mathcal{S}_{o,k}} \{\vec{s}_{n,k}\}_{j}\,,$$
where $\mathcal{S}_{o,k}$ is the set of all states contributing to the ion channel current, the ``open states''.

\subsection{Adaptive Timestepping}
\label{sec:adaptive-dt}
In order to accelerate the simulation in areas where there is little change to the dynamics, we choose an adaptive step size based on
\begin{equation}
  (\Delta t)_{n+1} = (\Delta t)_{n} \left(\frac{\Delta^{\rm tol}}{\sum_{k=1}^{M} N_k \norm{\vec{s}_{k,n+1} - \vec{s}_{k,n}}}\right)^{1/2}\,,
  \label{eq:dt-heuristic}
\end{equation}
for all $n$, where $\Delta^{\rm tol} \in \R^+$ is a measure for the allowed state change in between steps.
When the state changes too quickly in between time steps, the above heuristic will decrease $(\Delta t)_{n+1}$ and vice-versa.
In principle, it would be feasible to apply the adaptive timestepping to each ion channel type individually, however this would make current sampling and data synchronization between channels hard to realize, considering the ion concentration dependence of $H_k$.
Within this paper, we set $\Delta^{\rm tol} = 2 \cdot 10^{-7}$.

\subsection{Inverse Problem}
When regarding the cell model as a whole, the number of ion channels $N_k$ per type $k$ may be put into a configuration vector $\vec{N} := (N_1, ..., N_M)^T \in \N_0^M$ and the total simulated current $I$ sampled at measurement points $T_{\rm meas}$ can be expressed as a matrix-vector product
\begin{equation}
  \vec{I} = \sum_{k=1}^{M} N_k \vec{I_k} = R \vec{N}\,,
  \label{eq:matrix-formulation}
\end{equation}
where $R \in \R^{N_t \times M}$ is the matrix of all current measurements per channel type.

Given the individual ion channel type models' parameters, which we know from literature (cf. \Cref{table:channel-types}), the question that remains is how many channels there are of each type to fit the measurements.
This problem can be solved using a number of optimization approaches.

However, the formulation in \Cref{eq:matrix-formulation} also gives rise to a least-squares formulation, by projecting the measured current into the space of all individual channel currents.
More specifically, we want to find
\begin{equation}
  \vec{N}_{\rm opt} = \argmin_{\vec{N} \in \N_0^M} \frac{1}{2} \norm{R \vec{N} - \vec{I}_{\rm meas}}^2\,,
  \label{eq:minimization}
\end{equation}
with $\vec{I}_{\rm meas} \in \R^{N_t}$ the experimentally measured current.
The most important constraint here is that of integer non-negativity, $\vec{N}_{\rm opt} \in \N_0^M$, which makes this problem hard to solve directly.

The unconstrained least-squares problem could be solved very efficiently using a QR-decomposition of the current basis $R = Q_b R_b$, its solution would be $\vec{N}_{opt} = \lfloor R_b\inv Q_b^* \vec{I}_{\rm meas} \rceil \in \Z^M$.

\subsection{Formulation as a Quadratic Program}
Relaxing the integer condition on the solution, and letting $\vec{d} := \vec{I}_{\rm meas}$ for brevity, we can reformulate \Cref{eq:minimization},
$$\vec{N}_{\rm opt} \approx \argmin_{\vec{x} \in \R_+^M} f(\vec{x}) = \argmin_{\vec{x} \in \R_+^M} \sfrac{1}{2} \norm{R \vec{x} - \vec{d}}^2\,,$$
with cost function $f: \R^M \to \R^+$, which we manipulate to
\begin{align*}
  f(\vec{x}) & = \sfrac{1}{2} (R \vec{x} - \vec{d})^T (R \vec{x} - \vec{d})                                                          \\
             & = \sfrac{1}{2} \left(\vec{x}^T R^T R \vec{x} - \vec{x}^T R^T \vec{d} - \vec{d}^T R \vec{x} + \vec{d}^T \vec{d}\right) \\
             & = \sfrac{1}{2} \left(\vec{x}^T P \vec{x} + \vec{x}^T \vec{q} + \vec{q}^T \vec{x}\right)                               \\
             & = \sfrac{1}{2} \vec{x}^T P \vec{x} + \vec{q}^T \vec{x}
\end{align*}
where we let $P := R^T R \in \R^{M \times M}$ and $\vec{q} := -R^T \vec{d} \in \R^M$ and leave out the constant $\vec{d}^T \vec{d}$.
We can express the nonnegativity constraint $\vec{x} \ge \vec{0}$ as an equality constraint using a slack variable $\vec{s} \in \R_+^M$,
$$-\vec{x} + \vec{s} = \vec{0} \quad\Leftrightarrow\quad A \vec{x} + \vec{s} = \vec{b}\,,$$
where we set $A := -\mathds{1} \in \R^{M \times M}$ and $\vec{b} := \vec{0} \in \R^M$.
This leaves us with a constrained \textit{quadratic program},
\begin{align}
  \min_{\vec{x} in \R^M} & \frac{1}{2} \vec{x}^T P \vec{x} + \vec{q}^T \vec{x},     \\
  s.t.\;                 & A \vec{x} + \vec{s} = \vec{b}\,,\; \vec{s} \in \R_+^M\,.
\end{align}

We solve the quadratic problem in this exact form using Clarabel \cite{2024-clarabel}.
Note that in Clarabel notation, the slack variable is to be taken as an element of the nonnegativity cone.

The integer solution can then be obtained from rounding,
$$\vec{N}_{\rm opt} = \lfloor \vec{x} \rceil \in \N_0^M\,.$$

\begin{table}
  \caption{Ion Channel Types: Most fitting configuration and corresponding references.}
  \begin{tabular}{lrrl}
    \textbf{Channel Type} & \textbf{$\bm{N_k}$ \cite{2021-A549-model}} & \textbf{Our $\bm{N_k}$} & \textbf{Reference} \\
    \midrule
    Kv13                  & 22                                         & 13                      & \cite{kv13-1}      \\
    Kv31                  & 78                                         & 247                     & \cite{kv31}        \\
    Kv34                  & 5                                          & 10                      & \cite{kv34}        \\
    Kv71                  & 1350                                       & 1176                    & \cite{kv71}        \\
    KCa11                 & 40                                         & 38                      & \cite{kca11}       \\
    KCa31                 & 77                                         & 7                       & \cite{kca31-2}     \\
    Task1                 & 19                                         & 24                      & \cite{task1}       \\
    CRAC1                 & 200                                        & 188                     & \cite{cracm1}      \\
    TRPC6                 & 17                                         & 15                      & \cite{trpc6}       \\
    TRPV3                 & 12                                         & 10                      & \cite{trpv3}       \\
    CLC2                  & 13                                         & 234                     & \cite{clc2}        \\
  \end{tabular}
  \label{table:channel-types}
\end{table}

\subsection{Implementation and Usage}
The core simulation is implemented in the Rust programming language \cite{2014-rust}.
Each channel is represented as its own struct, implementing the \texttt{HasTransitionMatrix} trait.
The number of states per channel type is constant by design, allowing the compiler to optimize the instruction hierarchy and memory alignment of the simulation.
All individual ion channel types are then combined into the full \texttt{A549CancerCell} struct, whose most important function is \mintinline{rust}{cell.simulate();}.

For ease of use, we provide a Python interface to the simulation, ready to install via the Python Package Index (PyPI) as precompiled wheels for the most common platforms.

\begin{minted}{python}
  # pip install in-silico-cancer-cell
  from in_silico_cancer_cell import *

  # to evaluate on existing parameters
  cell = A549CancerCell.new()
  error = cell.evaluate(
    PatchClampProtocol.Ramp,
    CellPhase.G1
  )

  # to obtain the current basis matrix
  meas = PatchClampData.pyload(
    PatchClampProtocol.Activation,
    CellPhase.G0
  )
  p = ChannelCountsProblem.new(meas)
  p.precompute_single_channel_currents()
  single_channels = p.get_current_basis()
\end{minted}

\subsection{Live Simulation}
In addition to the programming interfaces, we also provide a live simulation web dashboard, available on \url{https://in-silico.hce.tugraz.at}. A screenshot is provided in \Cref{figure:screenshot}.
After choosing the voltage protocol and cell cycle phase, the dashboard runs the full simulation live, within the browser.
Computation time is well below a second for all protocol \& phase measurement configurations.
The dashboard shows the full current through the membrane at the top, compared to the measurement, while also breaking it down into individual channel type contributions.
Each channel type is shown as the evolution of its state space and resulting current over time.

\begin{figure}
  \includegraphics[width=\columnwidth]{../figures/results/voltage-protocol.pdf}
  \caption{The \textit{activation} voltage pulse protocol $V(t)$ used for the measurement and simulation. This is the voltage set across the A549 cell's membrane using a Patch-Clamp-System. The corresponding current is then recorded and depicted in \Cref{figure:full-simulation-current} along with its simulation result.}
  \label{figure:voltage-protocol}
\end{figure}

\section{Results}
\begin{figure}[ht]
  \includegraphics[width=\columnwidth]{../figures/results/full-simulation-current.pdf}
  \caption{The simulated and measured current response $I(t)$ across the membrane of an A549 cancer cell, given the \textit{activation} voltage protocol $V(t)$ in \Cref{figure:voltage-protocol}, recorded in the G0 phase of the cell cycle.}
  \label{figure:full-simulation-current}
\end{figure}
\begin{figure}
  \includegraphics[width=\columnwidth]{../figures/results/simulation-error.pdf}
  \caption{Pointwise squared error between the measured current $\vec{I}_{\rm meas}$ and the simulated current $\vec{I}$, showing potential systematic problems within the computational model's representation of the ion channels as compared to the experimental results.}
  \label{figure:simulation-error}
\end{figure}

\subsection{Adaptive Timestepping}
With the standard forward iteration approach, the simulation takes 412 million steps to simulate over all voltage protocols and cell cycle phases, while the adaptive timestepping only requires 9 million steps for the same configuration.
While keeping the same level of accuracy when matching with the experimental data, our adaptive timestepping method is therefore 45 times faster than the standard approach, on average.

\begin{figure}
  \includegraphics[width=\columnwidth]{../figures/results/dt-plot.pdf}
  \caption{Timestep $\Delta t$ throughout the duration of the simulation. $(\Delta t)_n$ is chosen in between simulation steps based on the heuristic given in \Cref{eq:dt-heuristic}. Red areas (small dt) indicate a high resolution within the simulation, blue areas indicate lower resolution (and therefore higher simulation speed).}
  \label{figure:dt-plot}
\end{figure}

\begin{figure}
  \includegraphics[width=\columnwidth]{../figures/results/delta-tolerance.pdf}
  \caption{Relative change of the average timestep $\Delta t$ (in blue), simulation runtime (in violet) and step acceptance rate (in green) when varying the delta tolerance $\Delta^{\rm tol}$ on a log-scale. All three quantities were normalized from their individual extent to $[0, 1]$. The most effective $\Delta^{\rm tol}$ is arguably on the order of $10^{-4}$.}
  \label{figure:delta-tolerance}
\end{figure}

\subsection{Inverse Problem}
We compare different optimization approaches based on their runtime and root mean square error,
$$\Delta = \sqrt{\sfrac{1}{N_t} {\textstyle \sum_{j=1}^{N_t}} (I_j - I_{{\rm meas}, j})^2} = \frac{\norm{\vec{I} - \vec{I}_{\rm meas}}}{\sqrt{N_t}}\,.$$

\begin{table*}[ht]
  \centering
  \caption{Comparison of Optimization Approaches}
  \begin{tabular}{llrr}
    \textbf{Algorithm}                          & \textbf{Abbreviation} & \textbf{Runtime} / s & \textbf{RMSE} / pA \\
    \midrule
    Particle Swarm Optimization                 & PSO                   & 22.5                 & 27.691             \\
    Gradient Descent + More Thuente             & GD                    & 18.9                 & 32.342             \\
    Limited-Memory BFGS + Hager Zhang           & LBFGS                 & 4.8                  & 32.196             \\
    Non-Negative Least Squares \cite{1997-nnls} & NNLS                  & 0.318                & 28.00              \\
    Quadratic Program                           & QP                    & 0.018                & 28.13              \\
  \end{tabular}
  \label{table:optimization-comparison}
\end{table*}

\begin{figure*}[ht]
  \includegraphics[width=\textwidth]{../figures/above-the-fold-screenshot.png}
  \caption{Screenshot of the live simulation dashboard, available here: \url{https://in-silico.hce.tugraz.at/}. This is the graphical user interface to the simulation written in Rust. The full simulation, completing within around 500ms, runs live in the browser using Rust's bindings to WebAssembly. One can choose the voltage protocol and cell cycle phase (G0/G1) in the menu on the left, while the full current and individual channel currents and internal states can be seen on the right. The remaining ion channel types have been cut off on this screenshot.}
  \label{figure:screenshot}
\end{figure*}

\section{Outlook}
Numerically, the stability of the simulation varies greatly with the time step state change tolerance $\Delta^{\rm tol}$, this could be improved using a higher-order integration scheme.
Regarding the simulation dashboard, there are still many adjustments that could improve and enable further usage perspectives.
Specifically, it would be effective to add visual handles for adjusting the number of channels per type and weigh their specific contributions, for instance.
