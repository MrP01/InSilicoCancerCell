Lung cancer is one of the most widespread pathologies worldwide and its mechanisms, specifically at the level of individual cells, are not well understood.
We improve on the A549 electrophysiological cancer cell model introduced in \cite{2021-A549-model,2024-calcium-channels}, combining numerical methods with an efficient implementation to reduce simulation time to a level where it is feasible for live interaction.
More specifically, we were able to accelerate the simulation with adaptive timestepping and a highly efficient implementation in the Rust programming language, while we also managed to approach the corresponding inverse problem using a quadratic program, solving it within milliseconds.
We introduce a visualisation approach of the entire model in the form of a live simulation dashboard available at \url{https://in-silico.hce.tugraz.at/} running directly in the browser.
The entire source code is freely available on GitHub and reusable through three different channels: the simulation interface (powered by compilation to WebAssembly), the Rust linkable library implementation and a Python package (simply run: \mintinline{text}{pip install in-silico-cancer-cell}).
Our aim behind a distribution in this way is to make the topic and simulation as accessible as possible.

% Through a reimplementation in Rust and numerical optimization approaches we were able to reduce the runtime of the A549 electrophysiological cancer cell model \cite{2021-A549-model} from X to Y.

% Please insert your abstract here. Remember that online
% systems rely heavily on the content of titles and abstracts to
% identify articles in electronic bibliographic databases and search
% engines. We ask you to take great care in preparing the abstract.
